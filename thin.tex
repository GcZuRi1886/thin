\documentclass{article}
\usepackage{amsmath}
\usepackage{amssymb}
\usepackage{graphicx}
\usepackage{float}
\usepackage[margin=1cm,footskip=0.25cm]{geometry}

\begin{document}
\begin{center}
    \textbf{\LARGE Theoretische Informatik}
\end{center}

\section*{Alphabete, Wörter und Sprachen}
\begin{minipage}[t]{0.45\textwidth}
    \subsection*{Alphabete}
    Ein \textbf{Alphabet} $\Sigma$ ist eine endliche, nichtleere Menge von Symbolen.
    
    \subsection*{Sprachen}
    Eine \textbf{Sprache} $L$ über einem Alphabet $\Sigma$ ist eine Menge von Wörtern, die aus Symbolen von $\Sigma$ bestehen. Eine Sprache kann endlich oder unendlich sein. Die leere Sprache wird mit $\emptyset$ bezeichnet.
\end{minipage}
\hfill
\begin{minipage}[t]{0.45\textwidth}
    \subsection*{Wörter}
    Ein \textbf{Wort} $w$ ist eine endliche Folge von Symbolen aus einem Alphabet $\Sigma$. Die Länge eines Wortes $w$ wird mit $|w|$ bezeichnet. Das leere Wort wird mit $\varepsilon$ dargestellt und hat die Länge 0.
    Die Menge aller Wörter über einem Alphabet $\Sigma$ wird mit $\Sigma^*$ bezeichnet (Kleenesche Hülle). 
\end{minipage}
\section*{Endliche Automaten}
\begin{minipage}[t]{0.45\textwidth}
    \subsection*{Deterministische endliche Automaten (DEA)}
    Ein DEA ist ein 5-Tupel $M = (Q, \Sigma, \delta, q_0, F)$, wobei:
    \begin{itemize}
        \item $Q$ eine endliche Menge von Zuständen ist,
        \item $\Sigma$ ein Alphabet ist,
        \item $\delta: Q \times \Sigma \to Q$ die Übergangsfunktion ist,
        \item $q_0 \in Q$ der Startzustand ist,
        \item $F \subseteq Q$ die Menge der Endzustände ist.
    \end{itemize}
    \textbf{Übergangsfunktion:} $\delta(q_0, a_1) = q_1$

    Ein Wort $w \in \Sigma^*$ wird akzeptiert, wenn es von $M$ verarbeitet wird und der Endzustand in $F$ liegt.
\end{minipage}
\hfill
\begin{minipage}[t]{0.45\textwidth}
    \subsection*{Nichtdeterministische endliche Automaten (NEA)}
    Ein NEA ist ähnlich aufgebaut, aber die Übergangsfunktion $\delta$ kann mehrere Zustände für ein Symbol zurückgeben: $\delta: Q \times \Sigma \to 2^Q$. Ein NEA akzeptiert ein Wort, wenn es mindestens einen Pfad gibt, der das Wort vollständig verarbeitet und in einem Endzustand endet.
\end{minipage}


\end{document}